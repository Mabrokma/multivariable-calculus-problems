\begin{question}
Suppose that a duck is swimming along a curve $x=3+2t$, $y=2-t^2$, while the water temperature is given by the formula $T = e^x(x + y)$. Find 
$\frac{dT}{dt}$ in two ways:
\begin{enumerate}
\item
by the chain rule and
\item
by expressing $T$ in terms of $t$ and differentiating.
\end{enumerate}
\end{question}

\begin{solution}
Note that we write $\frac{\p T}{\p x}$ and $\frac{\p T}{\p y}$, because $T(x,y)$ is a function of two variables. After subsituting $T(t)$ is a function of one variable and therefore we write $\frac{dT}{dt}$.
\begin{enumerate}
\item
By the chain rule the temperature of the duck is given by
\[
\frac{dT}{dt} = \frac{\p T}{\p x}\frac{dx}{dt} + \frac{\p T}{\p y} \frac{dy}{dt}\,.
\]
We have
\begin{align*}
\frac{\p T}{\p x} &= e^x(1 + x + y) &
\frac{\p T}{\p y} &= e^x\,,
\end{align*}
and after substitution
\begin{align*}
\frac{\p T}{\p x} 
&= e^{3+2t}\left(6 + 2t - t^2\right) &
\frac{\p T}{\p y} 
&= e^{3 + 2t}\,.
\end{align*}
The derivatives of $x(t)$ and $y(t)$ are
\begin{align*}
\frac{dx}{dt} &= 2 &
\frac{dy}{dt} &= -2t \,.
\end{align*}
Therefore
\begin{align*}
\frac{\p T}{\p x}\frac{dx}{dt} + \frac{\p T}{\p y} \frac{dy}{dt}
&= e^{3+2t}\left(12 + 2t - 2t^2\right)\,,
\end{align*}
by the chain rule.
\item
If we express $T$ as a function of $t$, we get
\[
T(t) = e^{3+2t}\left(5 + 2t - t^2\right)\,,
\]
and differentiating it leads to
\begin{align*}
\frac{dT}{dt} &= e^{3+2t} \left( 2 \left(5 + 2t - t^2\right) + 2 - 2t \right) \\
&= e^{3+2t} \left(12 + 2t - 2t^2 \right)\,.
\end{align*}
We see that both method lead to the same result.
\end{enumerate}
\end{solution}

\begin{question}
Suppose that a bird flies along a helical curve $x=2 \cos t$, $y = 2 \sin t$, $z=3t$. The bird suddenly encounters a weather front, so that the barometric pressure is varying rather wildly from point to point as $P(x,y,z) = \frac 3{10}\frac{x^2 z}{y} \,\textrm{atm}$.
\begin{enumerate}
\item
Use the chain rule to determine how the pressure is changing at $t = \frac \pi 4 \,\textrm{min}$.
\item
Using the linear approximation, determine the approximate pressure at $t=\frac \pi 4 + 0.01 \,\textrm{min}$.
\end{enumerate}
\end{question}

\begin{solution}
Note that we can ignore the physical units for most of the calculations, but we will use them to state the answer.
\begin{enumerate}
\item
The partial derivatives of $P(x,y,z)$ are
\begin{align*}
\frac{\p P}{\p x} &= \frac 35 \frac{xz}y &
\frac{\p P}{\p y} &= -\frac 3{10} \frac{x^2z}{y^2} &
\frac{\p P}{\p z} &= \frac 3{10} \frac{x^2}y\,,
\end{align*}
and the derivative of the bird's trajectory is
\begin{align*}
\frac{d x}{x t} &= -2 \sin t &
\frac{d y}{x t} &= 2 \cos t &
\frac{d z}{x t} &= 3 \,.
\end{align*}
The chain rule now gives
\begin{align*}
\frac{dP}{dt} &= \frac{9t}5 \frac {\cos t}{\sin t} \cdot (-2) \sin t
- \frac {9t}{10} \frac{\cos^2 t}{\sin^2 t} \cdot 2 \cos t
+ \frac 35 \frac{\cos^2t}{\sin t} \cdot 3 \\
&= -\frac{18t}5 \cos t - \frac{9t}{5} \frac{\cos^3 t}{\sin^2 t}
+ \frac 95 \frac{\cos^2t}{\sin t}\,.
\end{align*}
Evaluated at $t=\frac \pi 4$ we obtain
\[
\frac{dP}{dt}\left(\frac \pi 4\right) = -\frac{9\sqrt{2}}{10}\left(-\frac{3\pi}{4} + 1\right)
\approx - 1.726\ldots\,.
\]
Therefore at $\frac \pi 4 \,\textrm{min}$ the barometric pressure changes at a rate of $-1.726 \,\frac{\textrm{atm}}{\textrm{min}}$.

\item
First we evaluate
\[
P(\left(\frac \pi 4 \right) = \frac{9\pi\sqrt 2}{40} \approx 0.9996\ldots
\]
The linear approximation tells us that
\begin{align*}
P\left( \frac \pi 4 + 0.01 \right) &\approx P\left(\frac \pi 4 \right) 
+ \frac{dP}{dt}\left(\frac \pi 4 \right) \cdot 0.01 \\
&\approx 0.9996 - 1.726 \cdot 0.01 \approx 0.9823\,.
\end{align*}
\end{enumerate}
The problem is stated in physical units, so it is natural to enquire, whether the numbers make sense. At $t=\frac \pi 4 \,\textrm{min}$ the barometric pressure is $0.9996 \,\textrm{atm}$, which is average sea-level pressure. How much does pressure change over time? The highest recorded barometric presssure is about $1.08 \,\textrm{atm}$ and the lowest pressures occur during typhoons and are about $0.86 \,\textrm{atm}$. Hence a rate of change of $-1.726 \,\frac{\textrm{atm}}{\textrm{min}}$ would indicate an extremely rapid decrease.
\end{solution}

\begin{question}
Use the chain rule to determine an expression for the indicated derivative.
\begin{enumerate}
\item
Let $z=\sqrt{x^2+y^2}+2xy^2$, where $x$ and $y$ are functions of $u$. Find an expression for $\frac{dz}{du}$.
\item
Let $u =\sin\left(a + \ln b\right)$, where $a$ and $b$ are functions of $t$. Find an expression for $\frac{du}{dt}$.
\end{enumerate}
\end{question}

\begin{solution}
\begin{enumerate}
\item
The chain rule states that
\[
\frac{dz}{du} = \frac{\p z}{\p x}\frac{dx}{du}
+ \frac{\p z}{\p y}\frac{dy}{du}\,.
\]
The partial derivatives of $z=z(x,y)$ are
\begin{align*}
\frac{\p z}{\p x} &=
\frac{x}{\sqrt{x^2+y^2}} + 2y^2 &
\frac{\p z}{\p y} &=
\frac{y}{\sqrt{x^2+y^2}} + 4xy\,.
\end{align*}
Hence
\[
\frac{dz}{du} = \frac{1}{\sqrt{x^2+y^2}}
\left( x \frac{dx}{du} + y \frac{dy}{du} \right)
+ 2y^2 \frac{dx}{du} + 4xy \frac{dy}{du}\,.
\]
\item
The chain rule states that
\[
\frac{du}{dt} = \frac{\p u}{\p a}\frac{da}{dt}
+ \frac{\p u}{\p b}\frac{db}{dt}\,.
\]
The partial derivatives of $u=u(a,b)$ are
\begin{align*}
\frac{\p u}{\p a} &=
\cos(a + \ln b) &
\frac{\p u}{\p b} &=
\frac 1b \cos(a + \ln b) \,.
\end{align*}
Hence
\[
\frac{du}{dt} = \cos(a + \ln b) 
\left( \frac{da}{dt} + \frac 1b \frac{db}{dt} \right)\,.
\]
\end{enumerate}
\end{solution}

\begin{question}
For the following functions, calculate the derivative $\frac{d}{dt}\left(f \circ \bm{\si} \right)$ using the chain rule.
\begin{enumerate}
\item
$f(x,y) = (x^2+y^2)\ln\left(\sqrt{x^2+y^2}\right)$;
$\bm\si(t) = (e^t, e^{-t})$.
\item
$f(x,y) = xe^{x^2+y^2}$;
$\bm\si(t) = (t,-t)$.
\item
$f(x,y,z) = x+y^2+z^3$;
$\bm\si(t) = (\cos t, \sin t, t)$.
\item
$f(x,y,z) = e^{x-z}(y^2-x^2)$;
$\bm\si(t) = (t, e^t, t^2)$.
\end{enumerate}
\end{question}

\begin{solution}
We will write $\bm \si = (\si_1, \si_2)$ for the components of the curve $\bm \si$. Thus if $\bm \si(t) = (e^t, e^{-t})$, then $\si_1(t) = e^t$ and $\si_2(t) = e^{-t}$.
\begin{enumerate}
\item
The partial derivatives of $f$ are
\begin{align*}
f_x &= x \left( 2 \ln \sqrt{x^2+y^2} +\sqrt{x^2+y^2} \right) &
f_y &= y \left( 2 \ln \sqrt{x^2+y^2} +\sqrt{x^2+y^2} \right)\,,
\end{align*}
and
\[
\bm \si'(t) = \left(e^t, -e^{-t}\right)\,.
\]
Thus
\begin{align*}
\frac{d}{dt}\left(f \circ \bm \si\right) &=
e^t \left( 2 \ln \sqrt{ e^{2t} + e^{-2t}} + \sqrt{ e^{2t} + e^{-2t}} \right) \cdot e^t \\
&\qquad{} +
e^{-t} \left( 2 \ln \sqrt{ e^{2t} + e^{-2t}} + \sqrt{ e^{2t} + e^{-2t}} \right) \cdot \left(-e^{-t}\right) \\
&= \left(e^{2t}-e^{-2t}\right) \left(2\ln \sqrt{e^{2t}+e^{-2t}} + \sqrt{e^{2t}+e^{-2t}} \right)\,.
\end{align*}

\item
The partial derivatives of $f$ are
\begin{align*}
f_x &= (1 + 2x^2)e^{x^2+y^2} &
f_y &= 2xy e^{x^2+y^2}\,,
\end{align*}
and
\[
\bm \si'(t) = (1,-2)\,.
\]
Thus
\begin{align*}
\frac{d}{dt}\left(f \circ \bm \si\right) &=
 (1 + 2t^2)e^{2t^2} \cdot 1 - 2t^2 e^{2t^2} \cdot (-1) = \left( 1 + 4t^2 \right) e^{2t^2} \,.
\end{align*}

\item
The partial derivatives of $f$ are
\begin{align*}
f_x &= 1 &
f_y &= 2y &
f_z &= 3z^2\,,
\end{align*}
and
\[
\bm \si'(t) = (-\sin t, \cos t, 1)\,.
\]
Thus
\begin{align*}
\frac{d}{dt}\left(f \circ \bm \si\right) &=
 1 \cdot \left( - \sin t\right) + 2\sin t \cdot \cos t + 3t^2 \cdot 1
= -\sin t + 2 \sin t \cos t + 3t^2\,.
\end{align*}

\item
The partial derivatives of $f$ are
\begin{align*}
f_x &= e^{x-z}(y^2-2x-x^2) &
f_y &= 2ye^{x-z} &
f_z &= -e^{x-z}(y^2-x^2)\,.
\end{align*}
and
\[
\bm\si'(t) = (1, e^t, 2t)\,.
\]
Thus
\begin{align*}
\frac{d}{dt}\left(f \circ \bm \si\right) &=
 e^{t-t^2}\left(e^{2t}-2t-t^2\right) \cdot 1 + 2e^{t} e^{t-t^2} \cdot e^t
- e^{t-t^2} \left(e^{2t}-t^2\right) \cdot 2t \\
&= (3-2t)e^{3t-t^2} - (2t+t^2-2t^3)e^{t-t^2}\,.
\end{align*}
\end{enumerate}
\end{solution}

\begin{question}
In this problem we will use the chain rule to rederive results from single-variable calculus.
\begin{enumerate}
\item
Apply the chain rule to the function $u=xyz$, where $x=x(t)$, $y=y(t)$ and $z=z(t)$ are functions of $t$ to get a rule for differentiating a product of three functions of one variable.
\item
Use the chain rule and the function $f(y,z) = y^z$ to find $\displaystyle\frac{d}{dx}\left( x^x \right)$.
\end{enumerate}
\end{question}

\begin{solution}
\begin{enumerate}
\item
The chain rule gives
\[
\frac{du}{dt} = \frac{\p u}{\p x}\frac{dx}{dt}
+ \frac{\p u}{\p y}\frac{dy}{dt}
+ \frac{\p u}{\p z}\frac{dz}{dt} \,.
\]
The partial derivatives are
\begin{align*}
\frac{\p u}{\p x} &= yz &
\frac{\p u}{\p y} &= xz &
\frac{\p u}{\p z} &= xy\,,
\end{align*}
and hence
\[
\frac{du}{dt} = \frac{dx}{dt} yz
+ x \frac{dy}{dt} z
+ xy\frac{dz}{dt} \,,
\]
which is the product rule for a product of three factors. A more common form would be
\[
(fgh)' = f' gh + fg'h + fgh'\,.
\]
\item
If we consider $y(x)=x$ and $z(x)=x$ as functions of $x$, then
\[
x^x = f(x,x)\,,
\]
where $f(y,z) = y^z$. Using the chain rule we have
\[
\frac{d}{dx}\left( x^x \right) =
\frac{\p f}{\p y}\frac{dy}{dx} + 
\frac{\p f}{\p z}\frac{dz}{dx}\,.
\]
To compute the partial derivatives of $f$ we make the following observation: keeping $z$ fixed we can differentiate $y^z$ like a polynomial; keeping $y$ fixed we can write $y = e^{\ln y}$ and thus $y^z = e^{z \ln y}$. Therefore the partial derivatives of $f(y,z)$ are
\begin{align*}
\frac{\p f}{\p y} &= zy^{z-1} &
\frac{\p f}{\p z} &= y^z \ln y\,,
\end{align*}
and both $\frac{dy}{dx} = 1$ and $\frac{dz}{dx} = 1$. Thus
\[
\frac{d}{dx}\left( x^x \right) = x x^{x-1} + x^x \ln x = x^{x} \left( 1 + \ln x\right)\,.
\]
\end{enumerate}
\end{solution}

\begin{question}
Let $F(u,v)$ be a function of two variables. Suppose that $u = x+y$ and $v=xy$. Express $\frac{\p^2 F}{\p x \p y}$ in terms of $u$- and $v$-derivatives of $F$.
\end{question}

\begin{solution}
We first compute $\frac{\p F}{\p x}$ and $\frac{\p F}{\p y}$,
\begin{align*}
\frac{\p F}{\p x} &= \frac{\p F}{\p u} + y\frac{\p F}{\p v} &
\frac{\p F}{\p x} &= \frac{\p F}{\p u} + x\frac{\p F}{\p v}\,.
\end{align*}
Note that these formulas are valid for all functions $F(u,v)$. In particular we can replace $F$ by $\frac{\p F}{\p y}$,
\begin{align*}
\frac{\p}{\p x}\frac{\p F}{\p y} &=
\frac{\p}{\p x}\left(\frac{\p F}{\p u} + x\frac{\p F}{\p v}\right) \\
&= \frac{\p^2 F}{\p u^2} + y\frac{\p^2 F}{\p v \p u}
+ \frac{\p F}{\p v}
+ x \left( \frac{\p^2 F}{\p u \p v} + y\frac{\p^2 F}{\p v^2} \right) \\
&= \frac{\p F}{\p v} + \frac{\p^2 F}{\p u^2} 
+ (x + y) \frac{\p^2 F}{\p u \p v} + x y\frac{\p^2 F}{\p v^2}\,.
\end{align*}
When calculating $\frac{\p}{\p x} \left( x \frac{\p F}{\p v} \right)$ we had to apply the product rule. Differentiating the first factor gives $\frac{\p}{\p x}(x) = 1$ and for the second factor we use the result from above, which was obtained using the chain rule.
\end{solution}

\begin{question}
Let $u(x, y, z)$ be a function of three variables. Suppose that $x = pq^2 r$, $y = pq^3$ and $z = p^3$. Express $\frac{\p^2 u}{\p p \p r}$ in terms of $x$-, $y$- and $z$-derivatives of $u$.
\end{question}

\begin{solution}
We start with the first derivatives
\begin{align*}
\frac{\p u}{\p p} &= q^2 r \frac{\p u}{\p x} + q^3 \frac{\p u}{\p y}
+ 3p^2 \frac{\p u}{\p z} &
\frac{\p u}{\p r} &= pq^2 \frac{\p u}{\p x}\,.
\end{align*}
We do not need the $q$-derivative in this problem. Then
\begin{align*}
\frac{\p^2 u}{\p p \p r} &= \frac{\p}{\p p} \left(pq^2\right) \frac{\p u}{\p x} 
+ pq^2\left( q^2 r \frac{\p^2 u}{\p x^2} + q^3 \frac{\p^2 u}{\p y \p x}
+ 3p^2 \frac{\p^2 u}{\p z \p x} \right) \\
&= q^2 \frac{\p u}{\p x} 
+ pq^4 r \frac{\p^2 u}{\p x^2} + pq^5 \frac{\p^2 u}{\p x \p y}
+ 3p^3q^2 \frac{\p^2 u}{\p x \p z}\,.
\end{align*}
\end{solution}

\begin{question}
Cartesian (rectangular) coordinates $(x,y)$ of a point can be expressed in terms of polar coordinates $(r,\th)$ using the expressions
\begin{align*}
x &= r \cos \th &
y &= r \sin \th\,.
\end{align*}
Suppose that a function is given in terms of cartesian coordinates by $u=f(x,y)$. 
\begin{enumerate}
\item
Express $\frac{\p u}{\p r}$ and $\frac{\p u}{\p \th}$ in terms of
$\frac{\p u}{\p x}$ and $\frac{\p u}{\p y}$.
\item
Show that
\[
u_x^2 + u_y^2 = u_r^2 + \frac 1{r^2} u_\th^2\,.
\]
\item
Express
\[
u_{rr} + \frac 1r u_r + \frac 1{r^2} u_{\th\th}
\]
in cartesian coordinates.
\end{enumerate}
\end{question}

\begin{solution}
The derivative of $(x,y)$ with respect to $(r,\th)$ is
\begin{align*}
\frac{\p x}{\p r} &= \cos \th & \frac{\p x}{\p \th} &= -r \sin \th \\
\frac{\p y}{\p r} &= \sin \th & \frac{\p y}{\p \th} &= r \cos \th\,.
\end{align*}
\begin{enumerate}
\item
We can do this by simply applying the chain rule
\begin{align*}
\frac{\p u}{\p r} &= \cos \th \frac{\p u}{\p x} + \sin \th \frac{\p u}{\p y} \\
\frac{\p u}{\p \th} &= -r \sin\th \frac{\p u}{\p x} + r\cos \th \frac{\p u}{\p y}\,.
\end{align*}

\item
We start with the right hand side of the sought after identity,
\begin{align*}
u_r^2 + \frac 1{r^2} u_\th^2
&= \left(\cos \th\,u_x + \sin\th \, u_y\right)^2
+ \frac{1}{r^2} \left( -r\sin \th\, u_x + r \cos \th \,u_y\right)^2 \\
&= \left(\cos^2 \th + \sin^2 \th\right) u_x^2 + 
\left(\cos^2 \th + \sin^2 \th\right) u_y^2
= u_x^2 + u_y^2\,.
\end{align*}

Please note the difference between $u_x^2$ and $(u^2)_x$ and $u_{xx}$. By $u_x^2$ we mean the square of the function $u_x$, in other words $u_x^2 = (u_x)^2$. On the other hand $(u^2)_x$ is the $x$-derivative of the square $u^2$ and one has $(u^2)_x = 2uu_x$. Finally $u_{xx}$ is the second partial derivative. These are in general different functions.

\item
In order to express $u_{rr} + \frac 1r u_r + \frac{1}{r^2}u_{\th\th}$ in cartesian coordinates we need expressions for the second derivatives. First we have
\begin{align*} 
\frac{\p^2 u}{\p r^2}
&= \frac{\p}{\p r} \left( \cos \th\right) \frac{\p u}{\p x}
+ \cos \th \left( \cos \th \frac{\p^2 u}{\p x^2} + 
\sin \th \frac{\p^2 u}{\p y \p x} \right) \\
&\qquad + \frac{\p}{\p r} \left( \sin \th\right) \frac{\p u}{\p y}
+ \sin \th \left( \cos \th \frac{\p^2 u}{\p x \p y} + 
\sin \th \frac{\p^2 u}{\p y^2} \right) \\
&= \cos^2 \th \frac{\p^2 u}{\p x^2} + 2\sin \th \cos \th \frac{\p^2 u}{\p x \p y}
+ \sin^2 \th \frac{\p^2 u}{\p y^2}\,.
\end{align*}
Next is the second $\th$-derivative,
\begin{align*}
\frac{\p^2 u}{\p \th^2}
&= \frac{\p}{\p \th} \left( -r \sin \th\right) \frac{\p u}{\p x}
-r\sin \th \left(-r \sin \th \frac{\p^2 u}{\p x^2} + 
r \cos \th \frac{\p^2 u}{\p y \p x} \right) \\
&\qquad{} + \frac{\p}{\p \th} \left( r \cos \th\right) \frac{\p u}{\p y}
+ r \cos \th \left( -r\sin \th \frac{\p^2 u}{\p x \p y} + 
r \cos \th \frac{\p^2 u}{\p y^2} \right) \\
&= -r\cos \th \frac{\p u}{\p x} - r \sin \th \frac{\p u}{\p y}
+ r^2 \sin^2 \th \frac{\p^2 u}{\p x^2} - 2r^2 \sin \th \cos \th \frac{\p^2 u}{\p x \p y}
+ r^2 \cos^2 \th \frac{\p^2 u}{\p y^2}\,.
\end{align*}
The required sum is
\begin{align*}
u_{rr} + &\frac{1}{r} u_r + \frac{1}{r^2} u_{\th\th} = \\
&= \left( \cos^2 \th \, u_{xx} + 2 \sin \th \cos \th \, u_{xy}
+ \sin^2 \th \, u_{yy} \right)
+ \frac{1}{r} \left( \cos \th \, u_x + \sin \th \, u_y \right) \\
&\qquad{} + \frac{1}{r^2} \left( -r\cos \th \, u_x - r \sin \th \, u_y 
+ r^2 \cos^2 \th \, u_{xx} - 2r^2 \sin \th \cos \th \, u_{xy} 
+ r^2 \cos^2 \th \, u_{yy} \right) \\
&= \left( \cos^2 \th + \sin^2 \th \right) u_{xx} + 
\left( \cos^2 \th + \sin^2 \th \right) u_{yy}
= u_{xx} + u_{yy}\,.
\end{align*}
Therefore $u_{rr} + \frac 1r u_r + \frac{1}{r^2}u_{\th\th}$ becomes $u_{xx} + u_{yy}$ in cartesian coordinates.
\end{enumerate}
\end{solution}

\begin{question}
Compute the following derivative matrices and evaluate them at the given points.
\begin{enumerate}
\item
$\frac{\p(x,y)}{\p(u,v)}$, where
$x = u \sin v$, $y = e^{uv}$; evaluate at $(0,1)$.
\item
$\frac{\p(x,y,z)}{\p(r,\th,\ph)}$, where
$x = r\sin \ph \cos \th$, $y = r \sin \ph \sin \th$, $z = r \cos \ph$; evaluate at 
$\left(4, \frac\pi 4, \frac \pi 4\right)$.
\item
$\frac{\p(u,v)}{\p(x,y,z)}$, where
$u = xyz$, $v = x+y+z$; evaluate at $(1, 2, 3)$.
\end{enumerate}
\end{question}

\begin{solution} It is important to remember the order of partial derivatives in the derivative matrix.
\begin{enumerate}
\item
We have
\begin{align*}
\frac{\p(x,y)}{\p(u,v)} &=
\begin{pmatrix}
\sin v & u \cos v \\
v e^u & u e^v
\end{pmatrix} &
\frac{\p(x,y)}{\p(u,v)}(0,1) &=
\begin{pmatrix}
\sin 1 & 0 \\
1 & 0
\end{pmatrix}\,.
\end{align*}
\item
We have
\begin{align*}
\frac{\p(x,y,z)}{\p(r,\th,\ph)} &=
\begin{pmatrix}
\sin \ph \cos \th & -r \sin \ph \sin \th & r \cos \ph \cos \th \\
\sin \ph \sin \th & r \sin \ph \cos \th & r \cos \ph \sin \th \\
\cos \ph & 0 & -r \sin \ph
\end{pmatrix}\,.
\end{align*}
To evaluate at $\left(4, \frac \pi4, \frac \pi4\right)$ we use that $\sin \frac \pi4 = \frac{\sqrt 2}2$ and $\cos \frac \pi 4 = \frac{\sqrt 2}{2}$ and hence
\[
\frac{\p(x,y,z)}{\p(r,\th,\ph)}\left(4, \frac \pi 4, \frac \pi 4\right) =
\begin{pmatrix}
\tfrac 12 & -1 & 1 \\
\tfrac 12 & 1 & 1 \\
\tfrac{\sqrt 2}2 & 0 & -2
\end{pmatrix} \,.
\]

\item
We have
\begin{align*}
\frac{\p(u,v)}{\p(x,y,z)} &=
\begin{pmatrix}
yz & xz & xy \\
1 & 1 & 1
\end{pmatrix} &
\frac{\p(u,v)}{\p(x,y,z)}(1, 2, 3) &=
\begin{pmatrix}
6 & 3 & 2 \\
1 & 1 & 1
\end{pmatrix}\,.
\end{align*}
\end{enumerate}
\end{solution}

% \begin{question}
% Verify the general chain rule for each of the functions by following these steps:
% \begin{enumerate}[(i)]
% \item
% Compute the derivative matrices
% $\displaystyle\frac{\p(x,y)}{\p(t,s)}$ and
% $\displaystyle\frac{\p(u,v)}{\p(x,y)}$;
% \item
% Express $(u,v)$ in terms of $(t,s)$ and calculate
% $\displaystyle\frac{\p(u,v)}{\p(t,s)}$;
% \item
% Verify that the chain rule holds.
% \end{enumerate}
% Do this for the following functions:
% \begin{enumerate}
% \item
% $
% \begin{aligned}
% x &= t+s \\
% y &= t-s
% \end{aligned}
% \quad\text{ and }\quad
% \begin{aligned}
% u &= x^2 + y^2 \\
% v &= x^2 - y^2
% \end{aligned}
% \,.$

% \item
% $
% \begin{aligned}
% x &= t^2-s^2 \\
% y &= ts
% \end{aligned}
% \quad\text{ and }\quad
% \begin{aligned}
% u &= \sin(x+y) \\
% v &= \cos(x-y)
% \end{aligned}
% \,.$

% \item
% $
% \begin{aligned}
% x &= t^2+s^2 \\
% y &= t^2-s^2 \\
% z &= 2ts
% \end{aligned}
% \quad\text{ and }\quad
% \begin{aligned}
% u &= xy \\
% v &= yz \\
% w &= xz
% \end{aligned}
% $
% \end{enumerate}
% \end{question}

% \begin{solution}
% The chain rule states that
% \[
% \frac{\p(u,v)}{\p(t,s)} = \frac{\p(u,v)}{\p(x,y)}(t,s) \cdot \frac{\p(x,y)}{\p(t,s)}\,,
% \]
% where $\frac{\p(u,v)}{\p(x,y)}(t,s)$ denotes the Jacobi matrix of the functions $(u,v)$ with respect to $(x,y)$, but evaluated in the $(t,s)$ variables.

% \begin{enumerate}[(a)]
% \item
% Following the steps we have
% \begin{enumerate}[(i)]
% \item
% \begin{align*}
% \frac{\p (u,v)}{\p (x,y)} &=
% \begin{pmatrix}
% 2x & 2y \\
% 2x & -2y
% \end{pmatrix} &
% \frac{\p (u,v)}{\p (x,y)}(t,s) &=
% \begin{pmatrix}
% 2(t+s) & 2(t-s) \\
% 2(t+s) & -2(t-s)
% \end{pmatrix} &
% \frac{\p (x,y)}{\p (t,s)} &=
% \begin{pmatrix}
% 1 & 1 \\
% 1 & -1
% \end{pmatrix}
% \end{align*}
% \item
% We have
% \begin{align*}
% u &= (t+s)^2 + (t-s)^2 = 2t^2 + 2s^2\\
% v &= (t+s)^2 - (t-s)^2 = 4ts
% \end{align*}
% and the Jacobi matrix is
% \[
% \frac{\p (u,v)}{\p (t,s)} =
% \begin{pmatrix}
% 4t & 4s \\
% 4s & 4t
% \end{pmatrix}
% \]
% \item
% To verify the chain rule we need to compute the matrix product
% \[
% \frac{\p(u,v)}{\p(x,y)}(t,s) \cdot \frac{\p(x,y)}{\p(t,s)} = 
% \begin{pmatrix}
% 2(t+s) & 2(t-s) \\
% 2(t+s) & -2(t-s)
% \end{pmatrix} \cdot
% \begin{pmatrix}
% 1 & 1 \\
% 1 & -1
% \end{pmatrix}
% = \begin{pmatrix}
% 4t & 4s \\
% 4s & 4t
% \end{pmatrix}\,.
% \]
% \end{enumerate}

% \item
% Following the steps we have
% \begin{enumerate}[(i)]
% \item
% \begin{align*}
% \frac{\p (u,v)}{\p (x,y)} &\!=\!
% \begin{pmatrix}
% \cos(x+y) & \!\!\!\cos(x+y) \\
% -\sin(x-y) & \!\!\! \sin(x-y)
% \end{pmatrix} &\!\!\!\!
% \frac{\p (u,v)}{\p (x,y)}(t,s) &\!=\!
% \begin{pmatrix}
% \cos(t^2-s^2+ts) & \!\!\!\cos(t^2-s^2+ts) \\
% -\sin(t^2-s^2-ts) & \!\!\! \sin(t^2-s^2-ts)
% \end{pmatrix} &\!\!\!\!
% \frac{\p (x,y)}{\p (t,s)} &\!=\!
% \begin{pmatrix}
% 2t & -2s \\
% s & t
% \end{pmatrix}
% \end{align*}
% \item
% We have
% \begin{align*}
% u &= \sin(t^2-s^2+ts)\\
% v &= \cos(t^2-s^2-ts)
% \end{align*}
% and the Jacobi matrix is
% \[
% \frac{\p (u,v)}{\p (t,s)} =
% \begin{pmatrix}
% (2t+s) \cos(t^2-s^2+ts) & (-2s+t) \cos(t^2-s^2+ts) \\
% -(2t-s)\sin(t^2-s^2-ts) & -(-2s-t)\sin(t^2-s^2-ts)
% \end{pmatrix}
% \]
% \item
% To verify the chain rule we need to compute the matrix product
% \begin{align*}
% \frac{\p(u,v)}{\p(x,y)}(t,s) \cdot \frac{\p(x,y)}{\p(t,s)} &= 
% \begin{pmatrix}
% \cos(t^2-s^2+ts) & \cos(t^2-s^2+ts) \\
% -\sin(t^2-s^2-ts) & \sin(t^2-s^2-ts)
% \end{pmatrix} \cdot
% \begin{pmatrix}
% 2t & -2s \\
% s & t
% \end{pmatrix} \\ &=
% \begin{pmatrix}
% (2t+s) \cos(t^2-s^2+ts) & (-2s+t) \cos(t^2-s^2+ts) \\
% (-2t+s)\sin(t^2-s^2-ts) & (2s+t)\sin(t^2-s^2-ts)
% \end{pmatrix}\,.
% \end{align*}
% \end{enumerate}

% \item
% Following the steps we have
% \begin{enumerate}[(i)]
% \item
% \begin{align*}
% \frac{\p (u,v,w)}{\p (x,y, z)} &=
% \begin{pmatrix}
% y & x & 0 \\
% 0 & z & y \\
% z & 0 & x
% \end{pmatrix} &
% \frac{\p (u,v,w)}{\p (x,y,z)}(t,s) &=
% \begin{pmatrix}
% t^2-s^2 & t^2+s^2 & 0 \\
% 0 & 2ts & t^2-s^2 \\
% 2ts & 0 & t^2+s^2
% \end{pmatrix} &
% \frac{\p (x,y,z)}{\p (t,s)} &=
% \begin{pmatrix}
% 2t & 2s \\
% 2t & -2s \\
% 2s & 2t
% \end{pmatrix}
% \end{align*}
% \item
% We have
% \begin{align*}
% u &= (t^2+s^2)(t^2-s^2) = t^4 - s^4\\
% v &= 2t^3 s - 2ts^3 \\
% z &= 2t^3 s + 2ts^3
% \end{align*}
% and the Jacobi matrix is
% \[
% \frac{\p (u,v,w)}{\p (t,s)} =
% \begin{pmatrix}
% 4t^3 & -4s^3 \\
% 6t^2s - 2s^3 & 2t^3 - 6ts^2 \\
% 6t^2s + 2s^3 & 2t^3 + 6ts^2
% \end{pmatrix}
% \]
% \item
% To verify the chain rule we need to compute the matrix product
% \begin{align*}
% \frac{\p(u,v,w)}{\p(x,y,z)}(t,s) \cdot \frac{\p(x,y,z)}{\p(t,s)} &= 
% \begin{pmatrix}
% t^2-s^2 & t^2+s^2 & 0 \\
% 0 & 2ts & t^2-s^2 \\
% 2ts & 0 & t^2+s^2
% \end{pmatrix} \cdot
% \begin{pmatrix}
% 2t & 2s \\
% 2t & -2s \\
% 2s & 2t
% \end{pmatrix} \\&=
% \begin{pmatrix}
% 4t^3 & -4s^3 \\
% 6t^2s - 2s^3 & 2t^3 - 6ts^2 \\
% 6t^2s + 2s^3 & 2t^3 + 6ts^2
% \end{pmatrix} \,.
% \end{align*}
% \end{enumerate}
% \end{enumerate}
% \end{solution}

% \begin{question}
% Cartesian (rectangular) coordinates $(x,y)$ of a point can be expressed in terms of polar coordinates $(r,\th)$ using the expressions
% \begin{align*}
% x &= r \cos \th &
% y &= r \sin \th\,.
% \end{align*}
% Suppose that a function is given in terms of cartesian coordinates by $u=f(x,y)$. 
% \begin{enumerate}[(a)]
% \item
% Calculate the derivative matrix $\displaystyle\frac{\p(x,y)}{\p (r,\th)}$.
% \item
% Suppose that a function is given in terms of cartesian coordinates by $u=f(x,y)$. Express
% $\displaystyle\frac{\p u}{\p r}$ and $\displaystyle\frac{\p u}{\p \th}$ in terms of
% $\displaystyle\frac{\p u}{\p x}$ and $\displaystyle\frac{\p u}{\p y}$.
% \item
% Let $u=x^2+y^2$. Find $\displaystyle\frac{\p u}{\p r}$ and 
% $\displaystyle\frac{\p u}{\p \th}$.
% \item
% Express the polar coordinates $r$, $\th$ in terms of the cartesian coordinates $x$, $y$ and find the derivative matrix $\displaystyle\frac{\p(r,\th)}{\p (x,y)}$
% \end{enumerate}
% \end{question}

% \begin{solution}
% \begin{enumerate}[(a)]
% \item
% The derivative matrix is given by
% \[
% \frac{\p(x,y)}{\p (r,\th)} = 
% \begin{pmatrix}
% \cos \th & -r\sin \th \\
% \sin \th & r \cos \th
% \end{pmatrix}\,.
% \]
% \item
% We have
% \begin{align*}
% \frac{\p u}{\p r} &=
% \frac{\p u}{\p x} \frac{\p x}{\p r} + \frac{\p u}{\p y} \frac{\p y}{\p r}
% = \frac{\p u}{\p x} \cos \th + \frac{\p u}{\p y} \sin \th \\
% \frac{\p u}{\p \th} &=
% \frac{\p u}{\p x} \frac{\p x}{\p \th} + \frac{\p u}{\p y} \frac{\p y}{\p \th}
% = - r \frac{\p u}{\p x} \sin \th + r \frac{\p u}{\p y} \cos \th \,.
% \end{align*}
% \item
% We have
% \begin{align*}
% r &= \sqrt{x^2 + y^2} \\
% \th &= \arctan \frac y x\,,
% \end{align*}
% and so
% \[
% \frac{\p(x,y)}{\p (r,\th)} = 
% \begin{pmatrix}
% \displaystyle\frac{x}{\sqrt{x^2+y^2}} & \displaystyle\frac{y}{\sqrt{x^2+y^2}} \\
% \displaystyle\frac{-y}{x^2+y^2} & \displaystyle\frac{x}{x^2+y^2} 
% \end{pmatrix}\,.
% \]
% \end{enumerate}
% \end{solution}


\begin{question}
{\bfseries Differentiating under the integral.}
Under mild continuity restrictions, it is true that if
\[
F(x) = \int_a^b g(t,x) \ud t\,,
\]
then
$\displaystyle F'(x) = \int_a^b g_x(t,x) \ud t$.
Using this fact and the chain rule we can find the derivative of
\[
F(x) = \int_a^{f(x)} g(t,x) \ud t
\]
by letting
\[
G(u,x) = \int_a^u g(t,x) \ud t\,,
\]
where $u = f(x)$. Find the derivatives of the following functions:
\begin{enumerate}
\item
$\displaystyle
F(x) = \int_0^{x^2} \sqrt{t^4 + x^3} \ud t\,.
$
\item
$\displaystyle
F(x) = \int_{x}^{1} x \ln(t^2+x^2) \ud t\,.
$
\end{enumerate}
\end{question}

\begin{solution}
With $G(u,x)$ defined as above we have
\[
F(x) = G(f(x), x)
\]
and we can find the derivative of $F$ using the chain rule,
\[
F'(x) = G_u(f(x),x) f'(x) + G_x(f(x),x)\,.
\]
We know from the problem statement, that
\[
G_x(u,x) = \int_a^u g_x(t,x) \ud t\,,
\]
and what we need know is the derivative $G_u(u,x)$. But this we obtain from the fundamental theorem of calculus, which states that
\[
\frac{d}{du} \int_a^u g(t,x) \ud t = g(u,x)\,,
\]
or in our notation, $G_u(u,x) = g(u,x)$. It means that differentiating an integral with respect to the upper limit results in the function we are integrating. Putting it all together we obtain
\begin{align*}
F'(x) &= G_u(f(x),x) f'(x) + G_x(f(x),x) \\
&= g(f(x), x) f'(x) + \int_a^{f(x)} g_x(t,x) \ud t\,.
\end{align*}

Now we are ready to attempt the problems.
\begin{enumerate}
\item
$F(x) = \int_0^{x^2} \sqrt{t^4 + x^3} \ud t$. Then
\[
F'(x) = \sqrt{x^8 + x^3} \cdot 2x + \frac 32 \int_0^{x^2} \frac{x^2}{\sqrt{t^4+x^3}} \ud t\,.
\]
\item
We rewrite $F(x)$ as 
$F(x) = \int_{x}^{1} x \ln\left(t^2 + x^2\right) \ud t = -\int_1^{x} x \ln\left(t^2 + x^2\right) \ud t$. Then
\begin{align*}
F'(x) &= -x \ln \left(x^2 + x^2\right) - \int_1^x \frac{2x^2}{t^2 + x^2} \ud t \\
&= -x \ln \left(2x^2 \right) + \int_x^1 \frac{2x}{t^2 + x^2} \ud t\,.
\end{align*}
\end{enumerate}
\end{solution}

%%% Local Variables:
%%% TeX-master: "problems"
%%% End:
